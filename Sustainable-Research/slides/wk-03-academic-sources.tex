\documentclass[aspectratio=169]{beamer}
% ---- Minimal setup (LuaLaTeX) ----
\usepackage{fontspec}
\usepackage{luatexja}
\usepackage{luatexja-fontspec}
\ltjsetparameter{jacharrange={-9}}
\setmainfont{TeX Gyre Pagella}
\setsansfont{TeX Gyre Heros}
\setmainjfont{Noto Serif CJK JP}  % for Japanese/CJK in roman text
\setsansjfont{Noto Sans CJK JP}   % if you switch to \sffamily and still want CJK
\newfontfamily\emoji{Noto Color Emoji}[Renderer=Harfbuzz,Scale=MatchLowercase]
\newcommand{\e}[1]{{\emoji #1}}

\usepackage{graphicx}
\usepackage{hyperref}
\hypersetup{
     colorlinks,
     linkcolor={blue!50!black},
     citecolor={red!50!black},
     urlcolor={blue!80!black}
   }
\usepackage{booktabs}
\usepackage{microtype}

% --- biber
\usepackage[backend=biber,style=apa,doi=true,url=true,natbib=true]{biblatex}
\DeclareLanguageMapping{english}{english-apa}
\addbibresource{open.bib}

% ---- Beamer look ----
\usetheme{Boadilla}
\usecolortheme{seahorse}
\setbeamertemplate{navigation symbols}{}
\setbeamertemplate{itemize items}[circle]
\setbeamertemplate{itemize subitem}[triangle]
%\setbeamertemplate{footline}[frame number]

% ---- Section outline slide ----
\AtBeginSection[]{
  \begin{frame}
    \frametitle{Roadmap}
    \small
    \tableofcontents[currentsection, hideallsubsections]
  \end{frame}
}


% ---- Title metadata ----
\title[Academic Style \& Sources]{Open Knowledge for a Sustainable Future:\\ Research, Ethics, and Wikipedia}
\subtitle{Week 3  — Academic Style  \& Evaluating Sources}
\author[Bond \& Bednařík]{Francis Bond (\textbf{Academic}) \and Pavel Bednařík (Wiki)}
\institute[UPOL \& Wikimedia]{Palacký University Olomouc \quad | \quad Wikimedia ČR}
\date{14 October 2025}

\begin{document}

% =====================================================================
\begin{frame}
  \titlepage
\end{frame}

\begin{frame}
  \frametitle{Contents}
  \small
  \tableofcontents[hideallsubsections]
\end{frame}

% =====================================================================

\section{Why should we write?}

\begin{frame}{Purpose of Writing in the Humanities and Social Sciences}
\begin{itemize}
  \item Writing is both a \textbf{tool for thinking} and a \textbf{means of communication}.
  \item It helps clarify ideas, interpret texts, and contribute to scholarly conversations.
  \item Written work demonstrates:
    \begin{itemize}
      \item Understanding of key issues
      \item Ability to argue persuasively
      \item Awareness of disciplinary methods
    \end{itemize}
  \item Writing is a process of \emph{inquiry and reflection}, not merely reporting.
  \item Aim: to explore questions, not just provide answers.
\end{itemize}
\vfill
Partly based on \href{https://mlpp.pressbooks.pub/writinghandbook/chapter/chapter-1/}{Analyzing Texts, Taking Notes} \citep[Ch.~1]{writinghandbook}
\end{frame}



%------------------------------------------------------------
\begin{frame}{Humanities, Social Sciences, and Linguistics}
\begin{itemize}
  \item All three analyze human experience, culture, society, \emph{and} language.
  \item \textbf{Humanities:}
    \begin{itemize}
      \item Interpret texts, artworks, languages, and histories.
      \item Value close reading, interpretation, argumentation.
    \end{itemize}
  \item \textbf{Social Sciences:}
    \begin{itemize}
      \item Study human behavior, institutions, and systems.
      \item Employ observation, evidence, and models.
    \end{itemize}
  \item \textbf{Linguistics:} (bridging H/SS; both theoretical and empirical)
    \begin{itemize}
      \item \emph{General:} structure and use of language
      \item \emph{Variation \& change:} sociolinguistics, dialectology, historical linguistics.
      \item \emph{Mind \& processing:} psycholinguistics, neurolinguistics.
      \item \emph{Data \& methods:} corpora, fieldwork/elicitation, experiments, formal modeling.
      \item \emph{Technology \& applications:} computational linguistics/NLP, lexicography, language documentation.
    \end{itemize}
  \item Despite differences, all rely on:
    \begin{itemize}
      \item Critical analysis and evidence-based reasoning
      \item Clear written communication tailored to audience and genre
    \end{itemize}
\end{itemize}
\end{frame}

%------------------------------------------------------------
\begin{frame}{Writing as Inquiry}
\begin{itemize}
  \item Writing helps generate ideas and refine questions.
  \item Early drafts explore possibilities rather than finalize conclusions.
  \item Revision is discovery: each draft deepens understanding.
  \item Effective writers balance:
    \begin{itemize}
      \item \textbf{Open exploration} with
      \item \textbf{Focused argumentation}.
    \end{itemize}
  \item Thinking happens through writing, not before it.
\end{itemize}
\end{frame}

%------------------------------------------------------------
\begin{frame}{Academic Conversations}
\begin{itemize}
  \item Academic writing joins an ongoing \textbf{conversation of ideas}.
  \item You engage with others by:
    \begin{itemize}
      \item Quoting and analyzing sources
      \item Summarizing and synthesizing prior work
      \item Acknowledging different viewpoints
    \end{itemize}
  \item Essays must both \emph{respond to} and \emph{extend} these discussions.
  \item Citations show respect for others’ intellectual labor.
  \item Every essay adds a new voice to the dialogue.
\end{itemize}
\end{frame}

%------------------------------------------------------------
\begin{frame}{Developing a Question or Problem}
\begin{itemize}
  \item Essays begin with a focused, arguable question.
  \item Good questions are:
    \begin{itemize}
      \item Specific but open-ended
      \item Grounded in evidence
      \item Worth investigating
    \end{itemize}
  \item Avoid merely factual or yes/no questions.
  \item Examples:
    \begin{itemize}
      \item Weak: “Was Shakespeare popular?”
      \item Strong: “How did Shakespeare’s use of rhetoric shape his political commentary?”
    \end{itemize}
\end{itemize}
\end{frame}

%------------------------------------------------------------
\begin{frame}{Thesis and Argument}
\begin{itemize}
  \item The \textbf{thesis} presents your central claim.
  \item An argument:
    \begin{itemize}
      \item States a position clearly
      \item Provides reasons and evidence
      \item Anticipates counterarguments
    \end{itemize}
  \item Strong theses are \emph{debatable}, not descriptive.
  \item Structure builds logically from premise to conclusion.
  \item Each paragraph contributes to proving the thesis.
\end{itemize}
\end{frame}

%------------------------------------------------------------
\begin{frame}{Evidence and Interpretation}
\begin{itemize}
  \item Evidence supports reasoning; interpretation connects evidence to claims.
  \item Types of evidence:
    \begin{itemize}
      \item Textual quotation and analysis (Humanities)
      \item Data, case studies, and surveys (Social Sciences)
    \end{itemize}
  \item Avoid summary; explain significance.
  \item Analyze patterns and implications.
  \item Show how evidence leads logically to your conclusions.
\end{itemize}
\end{frame}

%------------------------------------------------------------
\begin{frame}{Audience Awareness}
\begin{itemize}
  \item Write for an informed but critical audience.
  \item Assume readers understand the basics but not your interpretation.
  \item Provide context and define specialized terms.
  \item Anticipate objections and address them respectfully.
  \item Maintain an academic tone—formal but engaging.
\end{itemize}
\end{frame}

%------------------------------------------------------------
\begin{frame}{Voice and Style}
\begin{itemize}
  \item Academic writing has a clear, confident voice.
  \item Strive for:
    \begin{itemize}
      \item Precision over ornamentation
      \item Clarity over complexity
      \item Variety in sentence structure
    \end{itemize}
  \item Avoid jargon unless necessary.
  \item Use active verbs and concise phrasing.
  \item Revision improves tone and flow.
  \item Essays need logical progression of ideas.
    \begin{itemize}
 \item Use outlines to maintain focus.
  \item Ensure every section supports the thesis.
    \end{itemize}
\end{itemize}
\end{frame}

% ------------------------------------------------------------
\begin{frame}{Revising and Editing}
\begin{itemize}
  \item Revision refines both ideas and expression.
  \item Strategies:
    \begin{itemize}
      \item Read aloud to test clarity
      \item Seek peer or instructor feedback
      \item Review argument flow
      \item Cut redundancy
    \end{itemize}
  \item Editing focuses on grammar, punctuation, and formatting.
  \item Always proofread before submission.
\end{itemize}
\end{frame}

%------------------------------------------------------------
\begin{frame}{Integrating Sources}
\begin{itemize}
  \item Use quotation, paraphrase, and summary effectively.
  \item Cite sources to:
    \begin{itemize}
      \item Credit others’ ideas
      \item Strengthen your credibility
      \item Help readers locate materials
    \end{itemize}
  \item Follow disciplinary citation style (MLA, APA, Chicago, etc.).
  \item Blend sources seamlessly with your own analysis.
\end{itemize}
\end{frame}

%------------------------------------------------------------
\begin{frame}{Academic Integrity}
\begin{itemize}
  \item Uphold honesty in research and writing.
  \item Avoid plagiarism by citing all borrowed ideas.
  \item Keep detailed notes on sources.
  \item Paraphrase thoughtfully; don’t just reword sentences.
  \item Academic trust depends on intellectual transparency.
\end{itemize}
\end{frame}

%------------------------------------------------------------
\begin{frame}{Becoming a Scholar}
\begin{itemize}
  \item Writing transforms students into active participants in knowledge creation.
  \item Scholars:
    \begin{itemize}
      \item Read critically
      \item Write reflectively
      \item Engage ethically with others’ ideas
    \end{itemize}
  \item Cultivate curiosity and persistence.
  \item Scholarship is a shared, evolving conversation.
\end{itemize}
\end{frame}

%------------------------------------------------------------
\begin{frame}{Summary}
\begin{itemize}
  \item Writing = Thinking + Communicating
  \item Humanities and social sciences differ in method but share core values.
  \item Essays are arguments supported by evidence.
  \item Revision and citation are essential scholarly habits.
  \item Goal: join the academic conversation with clarity and integrity.
\end{itemize}
\end{frame}

% =====================================================================
\subsection{Reading for Writing}
\begin{frame}{Reading for Writing}
\begin{itemize}
  \item Treat writing as a \textbf{thinking process} that begins with reading and note-taking.
  \item Approach every lecture, discussion, and reading as a \textbf{text} to analyze.
  \item Ask questions continually; \textbf{do not read passively}.
  \item Build habits that connect reading notes to future \textbf{essay arguments}.
  \item Aim to understand \emph{how} a “verbal contraption” works, not just what it says.
  \end{itemize}

\vfill
Partly based on \href{https://mlpp.pressbooks.pub/writinghandbook/chapter/chapter-1/}{Analyzing Texts, Taking Notes} \citep[Ch.~1]{writinghandbook}

  
\end{frame}

\begin{frame}{What Counts as a “Text”?}
\begin{itemize}
  \item Any statement encountered in class: readings, \textbf{lectures}, prepared discussions.
  \item Analyze texts \textbf{all the time}, not only when told to.
  \item Compare new material with prior readings, lectures, and beliefs.
  \item Write thoughts down—notes become the \textbf{foundation} of essays.
  \item Focus on both \textbf{content} and \textbf{method}: what it says and how it works.
\end{itemize}
\end{frame}

\begin{frame}{Active Reading Mindset}
\begin{itemize}
  \item Break the text into parts to see \textbf{purpose} and \textbf{mechanism}.
  \item Notice use of plot, imagery, symbolism, allusion (not just in literature).
  \item Recognize that nonfiction also deploys \textbf{language tools} strategically.
  \item Read to discover \textbf{patterns}, not only to collect facts.
  \item Remember: analyzing others’ writing prepares you to \textbf{write interpretively}.
\end{itemize}
\end{frame}

% \begin{frame}{Note-Taking as the Start of Writing}
% \begin{itemize}
%   \item Lecture notes record \textbf{arguments}, not just facts.
%   \item Identify the lecture’s \textbf{central question} or idea (check titles/syllabus).
%   \item After class, condense to a \textbf{one–two sentence} theme.
%   \item Ask for clarification when stumped—\textbf{inquiry} is part of the process.
%   \item Treat reading notes as \textbf{drafting} toward your essay.
% \end{itemize}
% \end{frame}

% \begin{frame}{From Lecture to Thesis-Ready Notes}
% \begin{itemize}
%   \item Extract key claims and \textbf{supporting reasons}.
%   \item Track examples and how they \textbf{function} in the argument.
%   \item Mark uncertainties and \textbf{counterpoints} raised implicitly.
%   \item Cross-reference with earlier lectures and readings.
%   \item Boil down to \textbf{testable claims} you might refine into a thesis.
% \end{itemize}
% \end{frame}

% \begin{frame}{Reading Notes: Facts \& Interpretations}
% \begin{itemize}
%   \item Record who/what/when/where/how—and proposed \textbf{why}.
%   \item Expect blurry borders between \textbf{fact} and \textbf{interpretation}.
%   \item Annotate margins with questions, connections, and \textbf{hypotheses}.
%   \item Compare with other texts, lectures, and your interests.
%   \item Treat notes as a \textbf{dialogue} with the text before discussion.
% \end{itemize}
% \end{frame}

% \begin{frame}{Arriving Prepared for Discussion}
% \begin{itemize}
%   \item Underline passages; write focused \textbf{questions} and comments.
%   \item Bring tentative \textbf{interpretations} and be ready to revise them.
%   \item Participation improves when you are an \textbf{active} reader.
%   \item Group discussion adds \textbf{perspectives} that approach “truth.”
%   \item Use discussion to test and \textbf{sharpen} your emerging claims.
% \end{itemize}
% \end{frame}

\begin{frame}{General Questions to Drive Analysis}
\begin{itemize}
  \item What confuses you? What needs \textbf{clarification}?
  \item Which claims are most \textbf{central} to the text’s project?
  \item How do structure and language \textbf{support} those claims?
  \item What assumptions or \textbf{premises} are in play?
  \item How does this text relate to other course materials and \textbf{your concerns}?
\end{itemize}
\end{frame}

\begin{frame}{For Fiction (Mostly)}
\begin{itemize}
  \item \textbf{Narrator}: who tells the story? reliable/unreliable/biased?
  \item \textbf{Setting \& tone}: what senses and emotions are evoked?
  \item \textbf{Characters}: motivations, alignments, identification cues.
  \item \textbf{Language/diction}: level and implications.
  \item \textbf{Plot/structure}: problems, challenges, archetypes.
  \item \textbf{Images/motifs}: repetitions, metaphors, patterns.
  \item \textbf{Ending}: what resolves? why end \emph{there}?
\end{itemize}
\end{frame}

\begin{frame}{For Nonfiction (Mostly)}
\begin{itemize}
  \item \textbf{Author}: background and qualifications.
  \item \textbf{Audience}: allies, opponents, or neutral readers?
  \item \textbf{Intention}: explanatory, polemical, celebratory—\emph{why} written?
  \item \textbf{Structure}: how is the argument organized?
  \item \textbf{Appeals}: logic vs. emotion; what types of arguments are used?
\end{itemize}
\end{frame}

\begin{frame}{On Arguments: Classical Roots}
\begin{itemize}
  \item Aristotle analyzed features of argument still relevant today.
  \item Logic is central to many, but not all, arguments.
  \item A \textbf{syllogism} shows how accepted premises force a conclusion.
  \item Recognize that not all premises are \textbf{incontrovertible}.
  \item Much real-world reasoning yields \textbf{probable} rather than absolute conclusions.
\end{itemize}
\end{frame}

\begin{frame}{Deduction (Syllogism) in Brief}
\begin{itemize}
  \item Moves from accepted premises to a \textbf{necessary} conclusion.
  \item If premises hold, disputing the conclusion is \textbf{illogical}.
  \item Useful when shared facts/definitions exist.
  \item Limits: debates often target the \textbf{premises} themselves.
  \item Practice: state premises explicitly; test their \textbf{soundness}.
\end{itemize}
\end{frame}

\begin{frame}{Induction}
\begin{itemize}
  \item Starts from observations/data and infers a \textbf{generalization}.
  \item Conclusions are \textbf{tentative} (we never observe everything).
  \item Science frames even strong theories as \textbf{revisable}.
  \item Good induction “\textbf{follows the data}.”
  \item Beware overreach; match claim strength to \textbf{evidence}.
\end{itemize}
\end{frame}

\begin{frame}{Narrative as Argument}
\begin{itemize}
  \item Stories and anecdotes can \textbf{persuade} by identification.
  \item History blends data, concepts, and \textbf{narrative structure}.
  \item Narrative can sometimes substitute for data or axioms.
  \item The most powerful stories \textbf{engage emotion}.
  \item Ask why a writer turns to story—what \textbf{work} is narrative doing?
\end{itemize}
\end{frame}

\begin{frame}{Reason, Emotion, and Premises}
\begin{itemize}
  \item Distinguish appeals to \textbf{reason} vs. \textbf{emotion}.
  \item Identify the trail from premises to conclusion.
  \item Test premises for \textbf{assumptions}, generality, and evidence.
  \item Map where uncertainty lies: data, inference, or \textbf{values}.
  \item Use your notes to plan a balanced, \textbf{well-supported} response.
\end{itemize}
\end{frame}

\begin{frame}{Quick Note-Taking Checklist}
\begin{itemize}
  \item Capture \textbf{central question} and main claims.
  \item Mark \textbf{evidence} and how it supports claims.
  \item Flag \textbf{key terms}, metaphors, and recurring motifs.
  \item Record \textbf{questions} and possible counterarguments.
  \item Synthesize into a \textbf{one–two sentence} takeaway for future drafting.
\end{itemize}
\end{frame}




% =====================================================================
\section{What is your message?}

\begin{frame}{Developing a Thesis — Chapter Overview}
\begin{itemize}
  \item The \textbf{thesis statement} is the backbone of any essay.
  \item It defines the argument and gives shape to analysis and evidence.
  \item A good thesis emerges from \textbf{questioning}, not from mere assertion.
  \item Writing itself helps \textbf{discover} the thesis.
  \item The thesis evolves through \textbf{drafting and revision}.
\end{itemize}

\vfill

Partly based on \href{https://mlpp.pressbooks.pub/writinghandbook/chapter/44/}{Creating a Thesis} \citep[Ch.~3]{writinghandbook}

\end{frame}

\begin{frame}{What Is a Thesis?}
\begin{itemize}
  \item A thesis is a claim that can be \textbf{defended with reasons and evidence}.
  \item It is neither a topic nor a fact, but an \textbf{interpretation}.
  \item Example:  
    \begin{itemize}
      \item Topic: “Women in Shakespeare.”  
      \item Thesis: “Shakespeare’s comedies use disguise to challenge gender norms.”
    \end{itemize}
  \item A thesis makes a promise to the reader about the essay’s direction.
\end{itemize}
\end{frame}

\begin{frame}{From Question to Argument}
\begin{itemize}
  \item Start with a genuine \textbf{question or problem}.
  \item Narrow broad curiosity into a focused inquiry.
  \item Ask “\emph{How? Why? So what?}” about your topic.
  \item As you read and write, your tentative answer becomes a \textbf{working thesis}.
  \item Revise the thesis as new evidence appears.
\end{itemize}
\end{frame}

\begin{frame}{Characteristics of a Strong Thesis}
\begin{itemize}
  \item \textbf{Debatable}: reasonable people could disagree.
  \item \textbf{Specific}: avoids vague generalities.
  \item \textbf{Focused}: manageable within the essay’s length.
  \item \textbf{Insightful}: reveals something not obvious.
  \item \textbf{Connected}: aligns with evidence and analysis.
  \item Weak vs.\ Strong Thesis Statements
\begin{itemize}
  \item Weak: announces a topic or restates a fact.  
    \begin{itemize}
      \item “This essay will discuss social media and teenagers.”
    \end{itemize}
  \item Strong: takes a clear, arguable stance.  
    \begin{itemize}
      \item “Social media intensifies teenage anxiety by rewarding performative identity.”
    \end{itemize}
  \item Strong theses provoke \textbf{“How?” and “Why?”} questions.
  \end{itemize}
\end{itemize}
\end{frame}

\begin{frame}{Thesis as a Map for the Reader}
\begin{itemize}
  \item The thesis signals what evidence matters.
  \item Each paragraph should support or test part of the claim.
  \item Readers use it to navigate your logic.
  \item Keep it visible—state it early and restate (refined) in the conclusion.
  \item Avoid burying the thesis in background or description.
\end{itemize}
\end{frame}

\begin{frame}{Refining Your Thesis}
\begin{itemize}
  \item Expect early theses to be \textbf{rough hypotheses}.
  \item Strengthen by:
    \begin{itemize}
      \item Clarifying key terms.
      \item Tightening scope.
      \item Checking consistency with evidence.
    \end{itemize}
  \item Ask peers to summarize your claim—does it match your intent?
  \item Revision turns a statement into a compelling argument.
\end{itemize}
\end{frame}

\begin{frame}{Types of Thesis Statements}
\begin{itemize}
  \item \textbf{Analytical}: interprets and explains evidence.  
        (e.g., “The novel critiques capitalism through its fragmented narration.”)
  \item \textbf{Expository}: explains a concept or process.  
        (useful for background essays)
  \item \textbf{Argumentative}: takes a position and justifies it.  
        (most common in humanities writing)
  \item Choose type according to essay’s purpose.
\end{itemize}
\end{frame}

\begin{frame}{Common Pitfalls}
\begin{itemize}
  \item Thesis too \textbf{broad} or too \textbf{narrow}.
  \item Merely \textbf{summarizes} instead of analyzing.
  \item Contains \textbf{multiple, unconnected} claims.
  \item Uses vague verbs: “shows,” “is about,” “explores.”
  \item Fails to anticipate \textbf{counterarguments}.
\end{itemize}
\end{frame}

%=== Writing Across Disciplines Slides ===%

\begin{frame}{Writing Across Disciplines}
\begin{itemize}
  \item \textbf{Humanities:} build an argument through interpretation.  
    \\ → \textit{Thesis-driven essay}: claim, evidence, counterargument.
  \item \textbf{Social Sciences:} explain social phenomena systematically.  
    \\ → \textit{IMRaD structure:} \textit{Introduction →  Methods →  Results →  Discussion}.
  \item \textbf{Linguistics:} mix of humanities and science.  
    \\ → \textit{Intro → Background → Data/Methods → Analysis → Discussion → Conclusion.}
  \item \textbf{Computer Science:} emphasize reproducibility and innovation.  
    \\ → \textit{Intro → Related Work → Method → Data → Experiments/Results → Analysis → Conclusion.}
  \item \textbf{Hybrids:} combine approaches (e.g., literature review + case study; policy analysis + recommendations).
\end{itemize}
\end{frame}

\begin{frame}{Purpose and Tone Across Disciplines}
\begin{tabular}{p{3cm}p{7cm}}
\textbf{Humanities} & Persuasive, interpretive, argument-driven.  
Focus on ideas and textual evidence. \\[0.5em]
\textbf{Social Sciences} & Empirical, objective tone.  
Focus on testing hypotheses, describing data. \\[0.5em]
\textbf{Linguistics} & Analytical, combining theory and data.  
Balances conceptual framing with empirical evidence. \\[0.5em]
\textbf{Computer Science} & Technical, concise, performance-oriented.  
Emphasis on algorithms, models, evaluation metrics.
\end{tabular}
\end{frame}

\begin{frame}{How Arguments Are Built}
\begin{itemize}
  \item \textbf{Humanities:} logic of persuasion → evidence supports an interpretation.
  \item \textbf{Social Sciences:} logic of proof → evidence tests a hypothesis.
  \item \textbf{Linguistics:} logic of demonstration → evidence shows a pattern or contrast.
  \item \textbf{Computer Science:} logic of replication → results must be reproducible.
  \item \textbf{All:} aim for clarity, coherence, and a sense of contribution.
\end{itemize}
\end{frame}

\begin{frame}{Encyclopedic Writing: Wikipedia Style}
\begin{itemize}
  \item \textbf{Purpose:} inform, not argue — summarize accepted knowledge.
  \item \textbf{Tone:} neutral, verifiable, non-original.
  \item \textbf{Structure:} topic-based, not narrative.  
    \\  \textit{Overview → Subtopics → References}.
    \\ \textit{Lead →  Body →  Appendices} \hfill \href{https://en.wikipedia.org/wiki/Wikipedia:Manual_of_Style/Layout}{Wikipedia:Manual of Style/Layout}
  \item \textbf{Comparison:}
    \begin{itemize}
      \item Unlike research writing, no new data or interpretation.  
      \item Like the introduction of an academic paper, it provides context and key sources.  
      \item Ideal for background reading, not for advancing claims.
    \end{itemize}
\end{itemize}
\end{frame}

\begin{frame}{Summary \& Takeaway}
\begin{itemize}
  \item A strong thesis:
    \begin{itemize}
      \item Arises from inquiry.
      \item Makes a specific, arguable claim.
      \item Guides structure and evidence.
    \end{itemize}
  \item Expect to revise it multiple times.
  \item Use feedback and reflection to sharpen the argument.
  \item Every paragraph should earn its place by advancing the thesis.
  \item Writing = continual \textbf{refinement of thought}.
  \item Different tasks have different goals
    \begin{itemize}
    \item You must asjust your writing style to fit the goal
    \item Different disciplines have different styles
    \end{itemize}
\end{itemize}
\end{frame}





% =====================================================================
\section{How can you convince people?}

\begin{frame}{From Topic to Argument}
\begin{itemize}
  \item Move from gathering ideas to \textbf{building a case}.
  \item Prefer \textbf{logical} appeals; use emotion sparingly and purposefully.
  \item Choose modes of reasoning suited to your materials.
  \item Keep conclusions \textbf{tentative yet confident}—acknowledge limits.
  \item Let structure make your thinking \textbf{followable} for readers.
  \end{itemize}
  \vfill
Partly based on \href{https://mlpp.pressbooks.pub/writinghandbook/chapter/ordering-evidence-building-an-argument/}{Ordering Evidence, Building an Argument} \citep[Ch.~4]{writinghandbook}
\end{frame}


\begin{frame}{Deduction and Induction in Practice}
\begin{itemize}
  \item \textbf{Deduction}: from accepted premises to a specific conclusion.
  \item[⇒] In real essays, premises are rarely beyond dispute—state them clearly.
  \item \textbf{Induction}: from specific data to generalization; always provisional.
  \item[⇒] Match claim strength to evidence quality and scope.
  \item Use both modes as needed; \textbf{hybrid} arguments are common.
\end{itemize}
\end{frame}

\begin{frame}{Architecture, Not Ornament}
  \begin{quote}
    Prose is architecture, not interior decoration, and the Baroque is over.
    \\
    \hfill \textnormal{Ernest Hemingway (1932) \textit{Death in the afternoon}}
  \end{quote}
\begin{itemize}
  \item Build on a \textbf{solid foundation}: thesis and linked reasons.
  \item \textbf{Form follows function}: structure should serve clarity.
  \item Mechanical scaffolding may be invisible, but must exist.
  \item Avoid random piles of points; design for \textbf{coherence}.
  \end{itemize}
 
\end{frame}

\begin{frame}{Finding Building Blocks}
\begin{itemize}
  \item Gather: \textbf{facts, quotations, data, prior interpretations}.
  \item Note how each item \textbf{functions} (example, counterexample, definition).
  \item Separate \textbf{summary} from \textbf{analysis} in notes.
  \item Track source details for citation and revisiting.
  \item Prune items that don’t advance the \textbf{central claim}.
\end{itemize}
\end{frame}

\begin{frame}{Outline: Before or After Drafting}
  \begin{itemize}
  \item Two approaches
    \begin{itemize}
    \item[A] sketch a \textbf{pre-outline} of controlling ideas (topic sentences).
    \item[B] \textbf{draft first}, then reverse-outline to reveal logic.
    \end{itemize}
  \item Either way, ensure the essay is \textbf{going somewhere}, not circling.
  \item Expect to \textbf{add/subtract/rearrange}—everything is tentative mid-process.
  \item Use outlines to test \textbf{progression} and \textbf{balance}.\\[2ex]
    
  \item I [FCB] normally write an outline and collect notes as I go along, then write prose at the end.  I write a rough introduction first, but revise it at the end, as I almost always change many details, \ldots
\end{itemize}
\end{frame}

\begin{frame}{The Working Model (Intro–Body–Conclusion)}
\begin{itemize}
  \item \textbf{Introduction}: hook interest; give only necessary context; state thesis.
  \item \textbf{Body}: organize supporting ideas into coherent paragraphs.
  \item \textbf{Transitions}: create \textbf{meaningful} links, avoid monotony.
  \item \textbf{Support}: back each assertion with \textbf{textual or data} evidence.
  \item \textbf{Conclusion}: reconnect claims; answer “\textit{so what?}”; mirror the intro.
\end{itemize}
\end{frame}

\begin{frame}{Paragraphs as Structural Beams}
\begin{itemize}
  \item Each paragraph advances \textbf{one} controlling idea.
  \item Start with a \textbf{topic sentence} tied to the thesis.
  \item Develop with \textbf{evidence + analysis}, not lists of facts.
  \item End by \textbf{linking forward} to the next step in the argument.
  \item Trim digressions; keep the \textbf{load-bearing} path visible.
\end{itemize}
\end{frame}

\begin{frame}{Sequencing and Emphasis}
\begin{itemize}
  \item Order points to create \textbf{momentum} (e.g., simple \textrightarrow{} complex).
  \item Front-load definitions; defer \textbf{nuances} until foundations are set.
  \item Place your strongest section where it has \textbf{maximum impact}.
  \item Use headings and transitions to signal \textbf{hierarchy} and shifts.
  \item Revisit sequence after drafting; \textbf{reshuffle} if clarity improves.
\end{itemize}
\end{frame}

\begin{frame}{Audience and Explicitness}
\begin{itemize}
  \item Define key terms; avoid assuming shared premises.
  \item Make \textbf{premises} and \textbf{purposes} explicit.
  \item Explain why evidence is \textbf{relevant}, not just that it exists.
  \item Balance brevity with the reader’s need for \textbf{orientation}.
  \item Prefer \textbf{readability} over flourish: clarity persuades.
\end{itemize}
\end{frame}

% \begin{frame}{Journey Metaphor: Guiding the Reader}
% \begin{itemize}
%   \item \textbf{Meet} in the introduction; \textbf{travel} together through the body.
%   \item Return in the conclusion to reflect on meaning and implications.
%   \item Different goals \textrightarrow{} different journeys; adapt the route.
%   \item If an unconventional structure \textbf{works}, keep it—just ensure wayfinding.
%   \item Solicit feedback to test whether the path \textbf{makes sense}.
% \end{itemize}
% \end{frame}

\begin{frame}{What you should aim for}
\begin{itemize}
  \item \textbf{Logical sequence}; momentum without stalls.
  \item \textbf{Smooth transitions}; visible through-line from thesis to conclusion.
  \item Claims \textbf{properly supported}; no orphan generalizations.
    \begin{itemize}
    \item Reliable sources clearly and correctly  cited
    \end{itemize}
  \item Overall emphasis aligns with the essay’s \textbf{central question}.
\end{itemize}
\end{frame}

% =====================================================================
\section{What are good sources?}
\begin{frame}{Evaluating Sources — Why It Matters}
\begin{itemize}
  \item Academic writing depends on \textbf{credible evidence}.
  \item Poor sources weaken even the best reasoning.
  \item Evaluating sources ensures:
    \begin{itemize}
      \item Accuracy and reliability
      \item Awareness of bias and limits
      \item Relevance to your argument
    \end{itemize}
  \item Evaluation is a \textbf{critical thinking skill}, not a checklist exercise.
  \end{itemize}

  \vfill
 Adapted from \href{https://wac.colostate.edu/resources/writing/guides/evaluating/}{Evaluating Sources} (WAC Clearinghouse, Colorado State University).  
\end{frame}
 

\begin{frame}{Purpose and Audience}
\begin{itemize}
  \item Ask: Why was this text created? For whom?
  \item Purposes may include:
    \begin{itemize}
      \item Informing or teaching
      \item Persuading or advocating
      \item Selling or entertaining
    \end{itemize}
  \item Identify intended audience: scholars, professionals, or the general public.
  \item Match the source’s aim with your own research goal.
\end{itemize}
\end{frame}

\begin{frame}{Author and Authority}
\begin{itemize}
  \item Who is the author, and what makes them credible?
  \item Check:
    \begin{itemize}
      \item Education and institutional affiliation
      \item Prior publications and expertise
      \item Reputation in the field
    \end{itemize}
  \item Anonymous or uncredentialed authors demand extra scrutiny.
  \item Authority may also stem from collective or institutional authorship.
\end{itemize}
\end{frame}

\begin{frame}{Publisher and Venue}
\begin{itemize}
  \item Who publishes or hosts the source?
  \item University presses and peer-reviewed journals usually signal quality control.
  \item For websites, assess the domain and hosting organization.
  \item Recognize potential \textbf{institutional bias} in think tanks, corporations, or advocacy groups.
  \item Prefer sources with transparent editorial oversight.
\end{itemize}
\end{frame}

\begin{frame}{Currency and Timeliness}
\begin{itemize}
  \item Consider when the source was written or updated.
  \item In fast-moving fields, information may age quickly.
  \item For historical or theoretical work, older sources may remain foundational.
  \item Look for revision dates, update logs, or newer editions.
  \item Always relate publication date to your topic’s context.
  \item Has the paper been \textbf{retracted}?
    \begin{itemize}
    \item Search in the official \href{ https://retractiondatabase.org}{Retraction Watch Database}
    \item Run a search by title, DOI, author, or journal.
    \item This is the most comprehensive, independent global database of retracted papers.
    \item It also notes the reason for retraction (e.g., plagiarism, data falsification, honest error).
    \end{itemize}
\end{itemize}
\end{frame}

\begin{frame}{Evidence and Support}
\begin{itemize}
  \item Reliable sources \textbf{show their work}.
  \item Ask:
    \begin{itemize}
      \item What kinds of evidence are used? (data, quotations, examples)
      \item Are sources cited and traceable?
      \item Is reasoning logical and transparent?
    \end{itemize}
  \item Unsupported claims or missing citations signal weakness.
  \item Cross-check evidence against other reputable works.
    \begin{itemize}
    \item Multiple sources are more reliable
    \end{itemize}
\end{itemize}
\end{frame}

\begin{frame}{Bias and Objectivity}
\begin{itemize}
  \item No source is completely neutral.
  \item Look for:
    \begin{itemize}
      \item Loaded language or emotional tone
      \item Selective omission of evidence
      \item Conflicts of interest or funding ties
    \end{itemize}
  \item Identify perspective; judge how it shapes interpretation.
  \item Acknowledge bias rather than ignoring it.
\end{itemize}
\end{frame}

\begin{frame}{Balance and Completeness}
\begin{itemize}
  \item Does the source present multiple viewpoints fairly?
  \item Recognize one-sided or partial presentations.
    \begin{itemize}
    \item If they cite themselves too much ($>25\%$) it is a bad sign
    \item If they only cite their colleaugues it is a bad sign
    \end{itemize}
  \item Check whether evidence contradicting the claim is addressed.
  \item Balanced sources strengthen your own credibility when cited.
  \item Even biased sources can be useful if analyzed critically.
\end{itemize}
\end{frame}

\begin{frame}{Relevance to Your Project}
\begin{itemize}
  \item Determine how the source connects to your research question.
  \item Directly relevant sources:
    \begin{itemize}
      \item Support or challenge your thesis
      \item Provide key evidence or theory
    \end{itemize}
  \item Peripheral sources may supply context or background.
  \item Avoid citing tangential material to inflate your bibliography.
\end{itemize}
\end{frame}

\begin{frame}{Primary vs.\ Secondary Sources}
\begin{itemize}
  \item \textbf{Primary:} original materials (texts, data, interviews, artifacts).
  \item \textbf{Secondary:} analysis, interpretation, commentary.
  \item  \textbf{Tertiary:} index or textual consolidation of  primary and secondary sources
  \item Choose according to purpose:
    \begin{itemize}
      \item Primary for direct evidence
      \item Secondary for framing and critique
      \item Tertiary for an overview
    \end{itemize}
  \item Distinguish between firsthand and filtered perspectives.
\end{itemize}
\end{frame}

\begin{frame}{Scholarly vs.\ Popular Sources}
\begin{itemize}
  \item \textbf{Scholarly:} peer-reviewed, technical, detailed references.
  \item \textbf{Popular:} general readership, journalistic style.
  \item Use scholarly works for evidence, popular for public context.
  \item Be cautious: some “grey literature” mixes the two.
  \item Evaluate tone, citations, and rigor to tell them apart.
\end{itemize}
\end{frame}

\begin{frame}{The Role of Peer Review}
\begin{itemize}
  \item Peer review adds accountability and expert evaluation.
  \item Check journal websites or  for peer-review status.
    \begin{itemize}
    \item  \href{https://www.scopus.com/sources}{Scopus Sources}:
 \\Includes review policy, coverage, and metrics.
\item \href{https://mjl.clarivate.com/}{Web of Science Master Journal List}: 
\\ Only indexed if peer-reviewed.
\item \href{https://ulrichsweb.serialssolutions.com}{Ulrichsweb Global Serials Directory}:
\\ look for \textit{Refereed: Yes}
    \end{itemize}
  \item Conference papers, reports, and blogs may lack external review.
  \item Non-reviewed sources can still inform background reading—use carefully.
  \item Note review processes in your evaluation notes.
    \begin{itemize}
    \item \textbf{Single-blind}: reviewer knows author
    \item \textbf{Double-blind}: neither reviewer nor author knows the other’s identity
    \item How many reviewers?
    \end{itemize}
\end{itemize}
\end{frame}
% \begin{frame}{Using the CRAAP Test Wisely}
% \begin{itemize}
%   \item CRAAP = Currency, Relevance, Authority, Accuracy, Purpose.
%   \item Handy memory aid but not exhaustive.
%   \item Combine with deeper questioning from WAC model.
%   \item Use it as a \textbf{starting framework}, not a mechanical rubric.
%   \item Always relate criteria to the \textbf{discipline’s expectations}.
% \end{itemize}
% \end{frame}

\begin{frame}{Synthesizing Multiple Sources}
\begin{itemize}
  \item Evaluation continues through comparison.
  \item Ask:
    \begin{itemize}
      \item How do sources agree or conflict?
      \item Which are most authoritative or current?
    \end{itemize}
  \item Synthesis reveals gaps and consensus in the field.
  \item Use evaluation to decide which sources to highlight or challenge.
\end{itemize}
\end{frame}

\begin{frame}{Checklist for Evaluating a Source}
\begin{itemize}
  \item Purpose and audience clearly stated?
  \item Author’s credentials and affiliations verifiable?
  \item Publisher or host credible and transparent?
  \item Evidence traceable and balanced?
  \item Date current enough for the topic?
  \item Bias recognized and context considered?
  \item Source relevant to your own argument?
\end{itemize}
\end{frame}


% =====================================================================
\section{How can I make it easy for my reader?}

\begin{frame}{Make the Information Accessible}

\begin{itemize}
  \item \e{🔍} \textbf{Identify the source clearly:}
    include author, year, title, publisher, and version or edition.
  \item \e{📍} \textbf{Pinpoint the exact location:}
    add chapter, section, or page number — especially for long works.
  \item \e{🌐} \textbf{Help readers find it quickly:}
    provide a persistent link (URL, DOI) or unique identifier (ISBN, dataset ID).
  \item \e{🧭} \textbf{Be consistent:}
    use one citation style throughout (APA, Chicago, etc.).
\end{itemize}

\end{frame}


\begin{frame}{Avoid Formatting or Mechanics Errors — Why It Matters}
\begin{itemize}
\item Formal citation styles encode a \textbf{“secret code”}
  \\ —tiny details convey location and source type.
  \item Using the community’s preferred style is a \textbf{“secret handshake”}: it signals you know the insider code.
  \item Attention to detail builds \textbf{credibility} with readers who value intellectual property and accuracy.
  \item Even in an era of easy keyword search, conventions still help readers \textbf{find and verify} sources quickly.
  \item Goal: demonstrate care and competence, not just avoid penalties.
    \begin{itemize}
    \item You should use software to help
    \end{itemize}
\end{itemize}
\end{frame}

\begin{frame}{The Code Behind Citation Styles (Context)}
\begin{itemize}
  \item Historically, information was hard to locate; styles evolved as \textbf{compressed wayfinding}.
  \item Visual cues (e.g., \textit{italics}/\underline{underlining} vs.\ “quotation marks”) signal \textbf{container vs.\ part}.
    \begin{itemize}
      \item \textit{Italic/underlined} titles: items bound into a book (containers).
      \item “Quoted/plain” titles: items inside a bound work (parts).
    \end{itemize}
  \item On screens, everything can look equal—but print-era cues still carry \textbf{meaning}.
  \item You may help evolve conventions later; for now, \textbf{learn and apply} the code.
\end{itemize}
\end{frame}

\begin{frame}{Pick the Right Style \& Identify Source Type}
\begin{itemize}
  \item Confirm the community preference: \textbf{MLA}, \textbf{APA}, \textbf{Chicago A/B} (new variants appear!).
  \item Determine what you’re citing:
    \begin{itemize}
      \item Book vs.\ journal article vs.\ whole website vs.\ a single post/section.
      \item Each type has \textbf{slightly different} required elements and order.
    \end{itemize}
  \item Differences usually make sense: books have titles/pages; tweets usually do not.
  \item Match the pattern to the \textbf{actual source features}.
  \item Avoid adding \textbf{unnecessary information}.
\end{itemize}
\end{frame}

\begin{frame}{Punctuation Around Quotations \& In-Text Citations}
\begin{itemize}
  \item Quotation punctuation follows \textbf{normal grammatical conventions}.
  \item In-text citations:
    \begin{itemize}
      \item \textbf{MLA}: \emph{no punctuation inside} the parentheses; sentence punctuation \emph{after} the citation.
      \item Other styles: may include \textbf{commas/abbreviations} inside the parentheses.
    \end{itemize}
  \item End-of-text entries (Works Cited/References) use style-specific patterns of
    \textbf{commas, colons, periods, italics, quotation marks}.
  \item Treat these patterns as part of the \textbf{code}, not decoration.
\end{itemize}
\end{frame}

\begin{frame}{Order of Information (Field-Sensitive Choices)}
\begin{itemize}
  \item Major difference: placement of \textbf{publication year}.
    \begin{itemize}
      \item \textbf{MLA}: year tends to appear \textbf{near the end}.
      \item \textbf{APA/others}: year appears \textbf{earlier}.
    \end{itemize}
  \item Rationale: recency matters more in fast-moving fields (e.g., AI) than in some literary analyses.
  \item Ensure elements are in the \textbf{correct sequence} for the style.
  \item Do not pad entries with \textbf{irrelevant} details.
\end{itemize}
\end{frame}

\begin{frame}{Capitalization \& Abbreviation Patterns}
\begin{itemize}
  \item Some styles favor \textbf{full capitalization} and spelled-out names/titles for formality.
  \item Others prefer \textbf{fewer capitalized words} and more \textbf{abbreviations} to speed reading.
  \item Apply title case vs.\ sentence case \textbf{as the style dictates}.
  \item Check consistent use of \textbf{standard abbreviations} (ed., trans., vol., no.).
  \item Consistency across entries is as important as correctness.
\end{itemize}
\end{frame}

\begin{frame}{Consistency is Non-Negotiable}
\begin{itemize}
  \item Pick a style and \textbf{stay with it}—don’t mix conventions.
  \item If you sometimes include the year and sometimes don’t, readers may suspect \textbf{incomplete acknowledgment}.
  \item For unusual sources (e.g., a deleted TikTok under a pseudonym), imitate the \textbf{closest established pattern}.
  \item Prioritize reader orientation: can they \textbf{find} what you cited?
  \item Keep a short \textbf{personal checklist} to enforce uniformity.
\end{itemize}
\end{frame}

\begin{frame}{Page Arrangement: Lists That Readers Can Scan}
\begin{itemize}
  \item Many end-of-text lists are \textbf{alphabetical} (by first element of the entry).
  \item Others are \textbf{chronological} or \textbf{numerical}—follow the assignment or venue.
  \item Use a \textbf{hanging indent} so lines after the first are indented—improves scanability.
  \item Maintain even spacing and \textbf{consistent} punctuation patterns across entries.
  \item Check that every in-text citation has a \textbf{matching} list entry (and vice versa).
\end{itemize}
\end{frame}

\begin{frame}{Tools Help—But You’re Still Responsible}
\begin{itemize}
\item Bibliography managers and library export tools can \textbf{misformat} elements.
  \begin{itemize}
  \item E.g. Nurril Hirfana binte Mohamed Noor, Suerya binte Sapuan and Francis Bond (2011)
    \\ cite as Nurril Hirfana, Suerya and Bond (2011) 
  \end{itemize}
  \item You must still \textbf{proofread} citations against the style rules.
  \item Learn enough of the code to \textbf{spot errors} quickly.
  \item Online forms may omit fields or guess wrong—\textbf{verify} and fill gaps.
  \item Working in a “\textbf{generation gap}” means tools + human judgment are both needed.
\end{itemize}
\end{frame}

\begin{frame}{Grace, Growth, and Credibility}
\begin{itemize}
  \item No one is born knowing citation mechanics; even strong writers make mistakes.
  \item The most important thing is that people can find the information.
  \item Errors \textbf{do not} imply bad faith—but accuracy \textbf{does} build trust.
  \item In communities that value intellectual property, detail work grants \textbf{power and credibility}.
  \item Over time, you join the discourse community that \textbf{evolves} conventions.
  \item For now: learn the handshake, apply it carefully, and \textbf{help readers}.
\end{itemize}
\end{frame}

% =====================================================================

\begin{frame}{Citing Works Not in English}
\begin{itemize}
  \item Scholarly writing often involves \textbf{sources in other languages}.
  \item Goals:
    \begin{itemize}
      \item Give credit to the original author.
      \item Help readers identify the work (even if they don’t read the language).
      \item Follow your citation style’s rules for \textbf{non-English titles}.
    \end{itemize}
  \item APA and most citation systems recommend:
    \begin{itemize}
      \item Keeping the original title in the source language.
      \item Optionally providing an \textbf{English translation in brackets}.
    \end{itemize}
  \item Transliterate non-Latin scripts if possible; retain diacritics accurately.
\end{itemize}
\end{frame}

\begin{frame}{General Principles (APA / biblatex)}
\begin{itemize}
  \item Use the author’s name in the script of publication (APA allows Latin transliteration).
  \item Give publication data exactly as printed (year, publisher, location).
  \item If the reader is unlikely to understand the title:
    \begin{itemize}
      \item Add a translation in square brackets:  
        \emph{Válka s Mloky [War with the Newts]}.
    \end{itemize}
  \item Don’t invent English titles — translate accurately but informally.
  \item If the work has an official English edition, you may cite that instead or alongside.
\end{itemize}
\end{frame}

\begin{frame}[fragile]{Example: Czech Source (Čapek, 1936)}
\small
\begin{verbatim}
@book{capek1936,
  author    = {Čapek, Karel},
  year      = {1936},
  title     = {Válka s Mloky [War with the Newts]},
  location  = {Praha},
  publisher = {Fr. Borový}
}
\end{verbatim}

\textbf{Text citation examples:}
\begin{itemize}
  \item \citet{capek1936} satirizes industrial modernity through the figure of the salamander.
  \item The allegory of human exploitation appears early in the narrative \citep{capek1936}.
\end{itemize}

\textbf{References entry (APA style):}
\begin{quote}
Čapek, K. (1936). \emph{Válka s Mloky [War with the Newts]}. Praha: Fr. Borový.
\end{quote}
\end{frame}

\begin{frame}[fragile]{Example: Japanese Source (芥川龍之介, 1918)}
\small
\begin{verbatim}
@book{akutagawa1918,
  author    = {芥川龍之介},
  year      = {1918},
  title     = {蜘蛛の糸 [Kumo no ito / The Spider's Thread]},
  publisher = {新潮社},
  location  = {東京}
}
\end{verbatim}

\textbf{Text citation examples:}
\begin{itemize}
  \item \citet{akutagawa1918} retells a Buddhist parable of redemption and failure.
  \item Compassion and egoism intertwine in “蜘蛛の糸” \citep{akutagawa1918}.
\end{itemize}

\textbf{References entry (APA style):}
\begin{quote}
芥川龍之介 (1918). \emph{蜘蛛の糸 [Kumo no ito / The Spider’s Thread]}. 東京: 新潮社.
\end{quote}
\end{frame}



\begin{frame}{When to Translate or Transliterate}
\begin{itemize}
  \item If the audience reads the language → keep original title only.
  \item If not → add translation in brackets after the original title.
  \item Transliteration (romaji, pinyin, etc.) helps with alphabetization and search.
  \item Example (Japanese romanization):
    \begin{quote}
    Akutagawa, Ryūnosuke. (1918). \emph{Kumo no ito [The Spider’s Thread]}.
    \end{quote}
  \item Always apply one consistent pattern for all non-English items.
\end{itemize}
\end{frame}

% \begin{frame}{In-Text Citation Behavior}
% \begin{itemize}
%   \item \textbf{biblatex} handles Unicode authors automatically.
%   \item In multilingual documents, citations appear correctly:
%     \begin{itemize}
%       \item \citet{capek1936} → Čapek (1936)
%       \item \citet{akutagawa1918} → 芥川龍之介 (1918)
%     \end{itemize}
%   \item Sorting by name or year still works with non-Latin scripts.
%   \item To force alphabetization by Latin transcription, add:  
%     \verb|sortname = {Capek, Karel}| or \verb|sortname = {Akutagawa, Ryunosuke}|.
% \end{itemize}
% \end{frame}

\begin{frame}{Checklist for Citing Non-English Sources}
\begin{itemize}
  \item Verify:
    \begin{itemize}
      \item Accurate author spelling and diacritics.
      \item Year, publisher, and city of publication.
      \item Correct script and optional translation.
    \end{itemize}
  \item Decide: original vs.\ translated title (or both).
%  \item Add \verb|sortname| for consistent alphabetical order.
%  \item Test output with \verb|\printbibliography|.
  \item Keep consistency across all non-English entries.
\end{itemize}
\end{frame}

\begin{frame}{Key Takeaways}
\begin{itemize}
  \item Cite foreign-language works with the same rigor as English sources.
  \item Use original titles + bracketed translations where helpful.
  \item Unicode and modern \textbf{biblatex} make multilingual citation smooth.
  \item Careful formatting demonstrates both linguistic and scholarly competence.
  \item Respect each language’s orthography while following APA consistency.
\end{itemize}
\end{frame}


% =====================================================================

\begin{frame}{Acknowledgements}

  \begin{itemize}
  \item The first sections were based on A Short Handbook for Writing Essays in the Humanities and Social Sciences \citep{writinghandbook}
  \item I also consulted \citet{Reid:2024} and \citet{Krause:2007}
  \item \textcite{openai-chatgpt-2025-03-14} was used to format the references, and generate a first draft of the slides with a prompt like
    \begin{flushleft}\texttt
       Please make me some slides, based on Chapter 1: https://mlpp.pressbooks.pub/writinghandbook/chapter/chapter-1/ \\

Make them in LaTeX, using luatex and biber, please make around 15 slides with at least 5 bullet points, use sub-lists where appropriate. \\
    \end{flushleft}
    
  \end{itemize}
  
\end{frame}




% =====================================================================
\begin{frame}[allowframebreaks]
\frametitle{References}
\printbibliography[heading=none]
\end{frame}

\end{document}
%%% Local Variables: 
%%% coding: utf-8
%%% mode: latex
%%% TeX-PDF-mode: t
%%% TeX-engine: luatex
%%% End: 

