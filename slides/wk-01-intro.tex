\documentclass[aspectratio=169]{beamer}

% ---- Minimal setup (LuaLaTeX) ----
\usepackage{fontspec}
\usepackage{luatexja}
\usepackage{luatexja-fontspec}
\ltjsetparameter{jacharrange={-9}}
\setmainfont{TeX Gyre Pagella}
\setsansfont{TeX Gyre Heros}
\setmainjfont{Noto Serif CJK JP}  % for Japanese/CJK in roman text
\setsansjfont{Noto Sans CJK JP}   % if you switch to \sffamily and still want CJK
\newfontfamily\emoji{Noto Color Emoji}[Renderer=Harfbuzz,Scale=MatchLowercase]
\newcommand{\e}[1]{{\emoji #1}}

\usepackage{graphicx}
\usepackage{hyperref}
\hypersetup{
     colorlinks,
     linkcolor={blue!50!black},
     citecolor={red!50!black},
     urlcolor={blue!80!black}
   }
\usepackage{booktabs}
\usepackage{microtype}

% --- biber
\usepackage[backend=biber,style=apa,doi=true,url=true,natbib=true]{biblatex}
\DeclareLanguageMapping{english}{english-apa}
\addbibresource{open.bib}

% ---- Beamer look ----
\usetheme{Boadilla}
\usecolortheme{seahorse}
\setbeamertemplate{navigation symbols}{}
\setbeamertemplate{itemize items}[circle]
\setbeamertemplate{itemize subitem}[triangle]
%\setbeamertemplate{footline}[frame number]

% ---- Section outline slide ----
\AtBeginSection[]{
  \begin{frame}
    \frametitle{Roadmap}
    \small
    \tableofcontents[currentsection, hideallsubsections]
  \end{frame}
}


% ---- Title metadata ----
\title[Open Knowledge --- Intro]{Open Knowledge for a Sustainable Future:\\ Research, Ethics, and Wikipedia}
\subtitle{Week 1 (Academic + Wiki) — Course Overview \& Framing}
\author[Bond \& Bednařík]{Francis Bond (Academic) \and Pavel Bednařík (Wiki)}
\institute[UPOL \& Wikimedia]{Palacký University Olomouc \quad | \quad Wikimedia ČR}
\date{23 September 2025}

\begin{document}

% =====================================================================
\begin{frame}
  \titlepage
\end{frame}

\begin{frame}
  \frametitle{Today}
  \small
  \tableofcontents[hideallsubsections]
\end{frame}

% =====================================================================
\section{Course Overview}

\begin{frame}{What This Course Is About}
\begin{itemize}
  \item Academic vs.\ encyclopedic writing (audience, voice, structure)
  \item Sustainable knowledge: FAIR / CARE, openness, ethics
  \item Practice both: (Academic) paper and (Wiki) article
  \item Learn transferrable skills: argument, evaluation, revision, collaboration
\end{itemize}
\end{frame}

\begin{frame}{Two Tracks, One Goal}
\begin{columns}[T,onlytextwidth]
\column{0.48\textwidth}
\textbf{(Academic)} Short paper (4-8 pp + refs)
\begin{itemize}
  \item Argument-driven, thesis-focused
  \item Synthesis and analysis
  \item Scholarly voice \& citation norms
  \item Writing part of a larger process
\end{itemize}

\column{0.48\textwidth}
\textbf{(Wiki)} Wikipedia article
\begin{itemize}
  \item NPOV, verifiability, no original research
  \item Clear structure; accessibility
  \item Community standards, consensus
\end{itemize}
\end{columns}

\vspace{4ex}
\begin{itemize}
  \item We \textbf{compare genres} to strengthen writing and judgment.
\end{itemize}
\end{frame}

\begin{frame}{Academic vs.\ Encyclopedic Writing (Quick Contrast)}
\begin{itemize}
  \item \textbf{Audience}: specialists vs.\ general public
  \item \textbf{Purpose}: advance/argue vs.\ summarize established knowledge
  \item \textbf{Voice}: hedged, theory-aware vs.\ neutral, accessible
  \item \textbf{Evidence}: engage literature vs.\ cite reliable secondary sources
  \item \textbf{Structure}: IMRaD/thesis vs.\ lede + sections
\end{itemize}
\end{frame}

\begin{frame}{Assessment Overview}
\begin{itemize}
\item \textbf{Academic paper (Francis)}: 4-8 pages + references; peer review; revision memo
  \begin{itemize}
  \item Review two articles
  \end{itemize}
  \item \textbf{Wikipedia article (Pavel)}: sandbox draft; sourcing; feedback; mainspace
  \begin{itemize}
  \item Comment on two articles
  \end{itemize}
  \item \textbf{Participation}: discussion, reflections
\end{itemize}
\end{frame}

% =====================================================================
\section{Why Academic Writing Matters}

\begin{frame}{Why It Matters}
\begin{itemize}
  \item It \textbf{helps thinking}: analysis, structure, precision
  \item It \textbf{builds knowledge}: claims + evidence + reasoning
  \item It \textbf{travels}: others can reuse, critique, extend
  \item It \textbf{signals credibility}: method, transparency, sources
  \item It \textbf{connects communities}: researchers, practitioners, public
  \item It \textbf{enables openness}: FAIR data and reusable prose
\end{itemize}
\end{frame}

\begin{frame}{Write for your Audience}
\begin{itemize}
\item \textbf{Discipline expectations}: terminology, typical arguments
\item \textbf{Background knowledge}: what can be assumed?
  \begin{itemize}
  \item Different disciplines are interested in different things (not always sensibly)
  \end{itemize}
\item \textbf{Background knowledge}: what can be assumed?
  \begin{itemize}
  \item You can assume some shared knowledge
  \item No need to explain everything
  \end{itemize}
\item \textbf{Motivation}: what problem are they trying to solve?
  \begin{itemize}
  \item Are you writing for basic research? \hfill understanding
  \item Applied research?  \hfill better practice
  \item Literary studies? \hfill better appreciation
  \item Fine arts? \hfill engage an audience
  \item \ldots{}
  \end{itemize}
\end{itemize}
\end{frame}

\begin{frame}{Define Your Purpose}
\begin{itemize}
  \item Explain, evaluate, compare, propose, synthesize?
  \item Purpose drives selection of evidence and structure
  \item One paper should have, one clear purpose 
    \begin{itemize}
    \item \textbf{Specific}: a claim you can support in a short paper
    \item \textbf{Contestable}: not a truism; invites argument
    \item \textbf{Roadmap}: hints at reasons/structure to come
    \end{itemize}
  \end{itemize}
\end{frame}

% =====================================================================
\section{Global Structure}

\begin{frame}{Common Structures}
\begin{itemize}
  \item Humanities: \textbf{thesis-driven essay}
  \item Social sciences: \textbf{IMRaD}
    \\  Intro → Methods → Results → Discussion → Conclusion.
  \item Linguistics: 
    \\   Intro → Background → Data/Methods → Analysis → Discussion → Conclusion
  \item Computer Science:
    \\ Intro → Related Work → Method → Data → Experiments/Results → Analysis → Conclusion
  \item Hybrids: literature review + case study; policy analysis + recommendations
\end{itemize}
\end{frame}

\begin{frame}{Start and end well}
  \begin{itemize}
  \item The Introduction sets the stage
\begin{itemize}
  \item Context \textrightarrow\ Problem \textrightarrow\ Question/Claim \textrightarrow\ Contribution \textrightarrow\ Roadmap
  \item Avoid: history lessons without focus; claims with no stakes
\end{itemize}
\item The Conclusion drives home the argument
  \begin{itemize}
  \item Answer the question; synthesize findings
  \item State implications, limits, next steps
  \item \textbf{Avoid}: repeating the intro; new evidence in the last paragraph
  \end{itemize}
\end{itemize}
\end{frame}

% =====================================================================
\section{Evidence and Sources (lecture 3)}
\begin{frame}{How to choose evidence}
  \begin{itemize}
  \item What can be trusted as evidence?
    \begin{itemize}
    \item Peer-reviewed articles; scholarly books
    \item Policy reports; official statistics; reputable NGOs
    \item Data (quantitative/qualitative), corpora, case studies
    \end{itemize}
  \item Evaluating Sources (Academic Lens)
    \begin{itemize}
    \item \textbf{Authority}: who wrote it? venue? peer review?
    \item \textbf{Recency/Relevance}: up-to-date, on-point
    \item \textbf{Method/Transparency}: can you inspect or replicate?
    \end{itemize}
  \end{itemize}
\end{frame}


\begin{frame}{Integrating Sources}
\begin{itemize}
  \item \textbf{Summarize}: key point in your words, with citation
  \item \textbf{Paraphrase}: reframe to serve your argument
  \item \textbf{Quote}: sparingly, when wording is crucial
  \item Always connect source to \textbf{your} claim
  \item Multiple sources are best
    \begin{itemize}
    \item Weave multiple sources to make a new point
    \item Compare/contrast findings, methods, assumptions
    \item Identify gaps, tensions, implications
    \end{itemize}
  \end{itemize}
\end{frame}

\begin{frame}{Multilingual Source Literacy}
\begin{itemize}
\item Often sources in other languages are ignored 
  \item When English is not the richest source: local journals, government docs, linguistic data, \ldots
  \item Beware translation bias; summarize fairly
  \item Cross-check facts across languages
\end{itemize}
\end{frame}
% =====================================================================
\section{Ethics, Integrity, Openness (lecture 5)}

\begin{frame}{Academic Integrity}
\begin{itemize}
\item Cite all sources (including data, images)
  \begin{itemize}
  \item Read what you cite, don't cite transitively (or acknowledge when you do)
  \item Cite in detail --- give page numbers (especially for books)
  \end{itemize}
  \item Proper paraphrase: change structure and wording, cite anyway
  \item Avoid patchwriting; keep notes disciplined
\end{itemize}
\end{frame}

\begin{frame}{Responsible Use of AI Tools}
\begin{itemize}
  \item AI can help brainstorm, outline, surface references
  \item \textbf{You} are responsible for accuracy, reasoning, and citation
  \item Disclose use where appropriate; never fabricate sources
\end{itemize}
\end{frame}

\begin{frame}{Sustainable Knowledge (Why Open?)}

This is more about data then writing, but still very important.
  
\begin{itemize}
  \item \textbf{FAIR}: Findable, Accessible, Interoperable, Reusable \cite{FAIR2016}
  \item \textbf{CARE}: Collective benefit, Authority, Responsibility, Ethics \cite{CARE2020}
  \item \textbf{Open Science} (UNESCO Recommendation) \cite{UNESCOOpenScience2021}
  \item \textbf{Knowledge equity} (Wikimedia 2030) \cite{KnowledgeEquity2030}
\end{itemize}
\end{frame}

% =====================================================================
\section{Process: Revision and Feedback (Lecture 7)}

\begin{frame}{Writing is Revising}
\begin{itemize}
\item Draft fast; revise for structure, then style
  \begin{itemize}
  \item \textbf{Content is more important than presentation}
  \end{itemize}
  \item Read aloud; reverse outline; get feedback early
  \item Track changes; keep a revision log
\end{itemize}
\end{frame}

\begin{frame}{Peer Review is part of the process}
  \begin{itemize}
  \item Helpful feedback
    \begin{itemize}
    \item Focus on \textbf{claims, evidence, logic}, not just grammar
    \item Ask: what is the thesis? is it supported? what is missing?
    \item Offer specific, actionable suggestions
    \item Don't be \href{https://www.youtube.com/watch?v=-VRBWLpYCPY}{reviewer three}\\
      \textbf{Do not:} Use a negative or dismissive tone; Ask for unreasonable
      revisions; Provide little or no constructive guidance; Let
      personal bias color the review.
    \end{itemize}
  \item Responding to Feedback
    \begin{itemize}
    \item Separate criticism into categories (structure, evidence, style)
    \item Decide: change, clarify, or justify (with reasons)
    \item Write a brief \textbf{revision memo}: what changed and why
    \end{itemize}
  \end{itemize}
\end{frame}

% =====================================================================
\section{General Advice: Common Mistakes \& Checklist}

\begin{frame}{Common Mistakes}
  \begin{itemize}
  \item   Topic Too Broad
    \begin{itemize}
    \item Fix: narrow to a focused question you can answer with available evidence
    \end{itemize}
  \item  Claim Without Evidence
    \begin{itemize}
    \item Fix: support with sources, data, or analysis; show how evidence bears on the claim
    \end{itemize}
    \item  Source Dump
      \begin{itemize}
      \item Fix: synthesize; explain why each source matters \emph{for your thesis}
      \end{itemize}
    \item  Structure Drift
      \begin{itemize}
      \item Fix: reverse outline; re-order sections to match argumentative flow
      \end{itemize}
    \item  Jargon Overload
      \begin{itemize}
      \item Fix: define key terms once; prefer plain English/Czech unless precision demands technical terms
      \end{itemize}
    \end{itemize}
\end{frame}

\begin{frame}{Pre-Submission Checklist}
\begin{itemize}
  \item Clear thesis in the introduction
  \item Section headings reflect argumentative moves
  \item Each paragraph has a topic sentence and a purpose
  \item Claims are cited; sources are integrated (not just quoted)
  \item Figures/tables are labeled and discussed
  \item Proofread for clarity, concision, coherence
  \item What Good Work Looks Like
    \begin{itemize}
    \item \textbf{Clarity}: readers always know the claim \& why it matters
    \item \textbf{Evidence}: claims anchored to credible sources
    \item \textbf{Ethics}: accurate attribution; respectful tone; UCoC compliance
    \item \textbf{Sustainability}: work that others can find, reuse, build upon
\end{itemize}
\end{itemize}
\end{frame}

\begin{frame}{A Minimal Model Outline}
\begin{itemize}
  \item \textbf{Abstract} (150--200 words)
  \item \textbf{Introduction}: context, question, thesis, roadmap
  \item \textbf{Body}: 2--3 sections building your case
  \item \textbf{Conclusion}: answer, implications, limits
  \item \textbf{References}: consistent style
\end{itemize}
\end{frame}

% =====================================================================
\section{(Wiki) Orientation \& Onboarding}

\begin{frame}{(Wiki) Ground Rules}
\begin{itemize}
  \item Wikipedia \textbf{Five pillars} \cite{wikipediaFivePillars}
  \item \textbf{UCoC}: Universal Code of Conduct \cite{WikimediaUCoC}
  \item \textbf{Programs \& Events Dashboard} enrollment \cite{WikiEduDashboard}
  \item Starter modules: editing basics, sourcing, plagiarism
\end{itemize}
\end{frame}

% =====================================================================
\begin{frame}[allowframebreaks]
\frametitle{References}
\printbibliography[heading=none]
\end{frame}

\end{document}
%%% Local Variables: 
%%% coding: utf-8
%%% mode: latex
%%% TeX-PDF-mode: t
%%% TeX-engine: luatex
%%% End: 

